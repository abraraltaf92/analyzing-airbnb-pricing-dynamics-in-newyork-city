\documentclass[12pt]{article}

% ---------------- Packages ----------------
\usepackage[margin=1in]{geometry}
\usepackage{setspace}
\usepackage{graphicx}
\usepackage{amsmath}
\usepackage{booktabs}
\usepackage{float}
\usepackage{hyperref}
\usepackage{caption}
\usepackage{subcaption}

\setstretch{1.3}

% ---------------- Title ----------------
\title{
\textbf{Analyzing Airbnb Pricing Dynamics in New York City} \\
\vspace{0.3cm}
\large Using Clustering and Regression Techniques
}

\author{
Abrar Altaf Lone \\
Data Mining Fall 2025 72958 \\
MS in Data Science \\
Pace University
}

\date{Dec 16 2025}

% ---------------- Document ----------------
\begin{document}

\maketitle
\thispagestyle{empty}
\newpage

% ---------------- Abstract ----------------
\begin{abstract}
This study analyzes Airbnb listings in New York City to understand pricing behavior
using data mining techniques. Exploratory data analysis, clustering, and regression
models are applied to identify pricing patterns and predict listing prices based on
location, demand, and availability characteristics. Results show that non-linear models
outperform linear baselines, highlighting the complexity of Airbnb pricing dynamics.
\end{abstract}

\newpage
\tableofcontents
\newpage

% ---------------- Introduction ----------------
\section{Introduction}
Airbnb pricing in New York City varies significantly due to factors such as location,
room type, demand, and regulatory constraints. Understanding these factors is important
for hosts, guests, and policymakers. This project applies data mining techniques to
explore pricing patterns and build predictive models using real-world Airbnb data.

% ---------------- Problem Statement ----------------
\section{Problem Statement}
The objective of this project is to analyze Airbnb listings in New York City to identify
pricing patterns, segment listings using clustering techniques, and predict listing prices
using supervised learning models.

% ---------------- Dataset Description ----------------
\section{Dataset Description}
The dataset used in this study is obtained from Inside Airbnb~\cite{insideairbnb} and contains summary-level
information on Airbnb listings in New York City. The file \texttt{listings.csv} includes
listing characteristics related to price, location, demand, and host activity.

After preprocessing, the dataset contains approximately 20,900 listings with 10
predictor variables and one target variable.

% ---------------- Data Preprocessing ----------------
\section{Data Preprocessing}

\subsection{Missing Values}
Listings with missing price values were removed, as price is the target variable.
Missing values in review-related features were imputed with zero to indicate no recent
review activity.

\subsection{Feature Selection}
Text-heavy attributes, identifiers, and metadata not directly related to pricing were
excluded to improve interpretability and reduce dimensionality.

\subsection{Target Transformation}
Price was log-transformed to reduce skewness and stabilize variance for regression modeling.

\subsection{Outlier Handling}
An interquartile range (IQR)–based approach was applied to the log-transformed price.
Approximately 1.8\% of extreme observations were removed, retaining realistic price ranges
between USD 16 and USD 1,500.

% ---------------- Exploratory Data Analysis ----------------
\section{Exploratory Data Analysis}

\subsection{Price Distribution}

\begin{figure}[H]
\centering
\includegraphics[width=0.75\textwidth]{figures/log_price_distribution.png}
\caption{Distribution of Log-Transformed Airbnb Prices}
\label{fig:logprice}
\end{figure}

Figure~\ref{fig:logprice} shows that the log-transformed price distribution is approximately
symmetric and unimodal, indicating that the transformation effectively reduced skewness.

\subsection{Price by Borough}

\begin{figure}[H]
\centering
\includegraphics[width=0.75\textwidth]{figures/price_by_borough.png}
\caption{Log-Transformed Airbnb Prices by Borough}
\label{fig:borough}
\end{figure}

Figure~\ref{fig:borough} highlights substantial pricing differences across boroughs.
Manhattan listings exhibit the highest median prices and greatest variability, while
the Bronx has the lowest median prices.

\subsection{Price by Room Type}

\begin{figure}[H]
\centering
\includegraphics[width=0.75\textwidth]{figures/price_by_room_type.png}
\caption{Log-Transformed Airbnb Prices by Room Type}
\label{fig:roomtype}
\end{figure}

Figure~\ref{fig:roomtype} shows that entire homes and hotel rooms command higher prices
than private and shared rooms, indicating room type as a major determinant of pricing.

\subsection{Correlation Analysis}

\begin{figure}[H]
\centering
\includegraphics[width=0.8\textwidth]{figures/correlation_matrix.png}
\caption{Correlation Matrix of Numeric Features}
\label{fig:corr}
\end{figure}

Figure~\ref{fig:corr} indicates weak linear correlations between most predictors and
price, motivating the use of multivariate and non-linear modeling techniques.

% ---------------- Methods ----------------
\section{Methods}

\subsection{Clustering}

\begin{figure}[H]
\centering
\includegraphics[width=0.7\textwidth]{figures/elbow_method.png}
\caption{Elbow Method for Selecting the Number of Clusters}
\label{fig:elbow}
\end{figure}

K-Means clustering was applied to standardized numeric features. As shown in
Figure~\ref{fig:elbow}, a noticeable reduction in marginal gains occurs beyond $k=5$,
which was selected as the optimal number of clusters. The silhouette score of approximately
0.33 indicates reasonable cluster separation.

\subsection{Regression Models}
Linear regression was used as a baseline model, followed by a Random Forest Regressor to
capture non-linear relationships and feature interactions.

% ---------------- Results ----------------
\section{Results}

\subsection{Clustering Results}
Five distinct clusters were identified, representing typical listings, budget listings,
high-demand listings, professional hosts, and regulation-driven long-term listings.

\begin{table}[H]
\centering
\caption{Cluster Summary Statistics}
\label{tab:cluster_summary}

\resizebox{\textwidth}{!}{%
\begin{tabular}{lcccccc}
\toprule
Cluster & Size & Avg Log Price & Avg Min Nights & Avg Reviews & Avg Reviews/Month & Avg Availability \\
\midrule
0 & 1029  & 6.06 & 31.24 & 0.17  & 0.01 & 224.99 \\
1 & 5873  & 4.92 & 25.08 & 26.86 & 0.64 & 115.70 \\
2 & 11910 & 5.00 & 28.80 & 16.59 & 0.38 & 321.48 \\
3 & 2068  & 5.22 & 9.49  & 204.43 & 4.26 & 236.69 \\
4 & 54    & 5.11 & 370.93 & 17.56 & 0.18 & 302.57 \\
\bottomrule
\end{tabular}%
}
\end{table}

Table~\ref{tab:cluster_summary} summarizes the characteristics of each cluster.
Clusters differ notably in terms of price, minimum stay requirements, review activity,
and availability. These differences enable meaningful interpretation of listing segments,
such as high-demand short-stay listings, long-term rental listings, and professionally
managed properties.

\subsection{Regression Results}
Table~\ref{tab:comparison} compares regression model performance.

\begin{table}[H]
\centering
\caption{Regression Model Performance}
\label{tab:comparison}
\begin{tabular}{lcc}
\toprule
Model & $R^2$ & RMSE (log price) \\
\midrule
Linear Regression & 0.50 & 0.55 \\
Random Forest Regressor & 0.72 & 0.41 \\
\bottomrule
\end{tabular}
\end{table}

Table~\ref{tab:comparison} shows that the Random Forest Regressor achieves substantially stronger
predictive performance than the linear regression baseline. The linear model explains about 50\% of
the variance in log-transformed prices ($R^2=0.50$), while the Random Forest improves explanatory power
to about 72\% ($R^2=0.72$) and reduces prediction error (RMSE from 0.55 to 0.41 on the log-price scale).
This improvement suggests that Airbnb pricing is not well-approximated by a purely linear relationship
and instead depends on non-linear effects and interactions among location, room type, demand indicators,
and availability constraints.

\subsection{Feature Importance Analysis}

\begin{figure}[H]
\centering
\includegraphics[width=0.75\textwidth]{figures/feature_importance.png}
\caption{Top 10 Feature Importances from the Random Forest Regressor}
\label{fig:feature_importance}
\end{figure}

Figure~\ref{fig:feature_importance} summarizes the most influential predictors in the Random Forest model.
Room type (especially entire home/apartment) emerges as a dominant driver of price, reflecting the premium
associated with full-unit rentals. Minimum night requirements and host listing count also contribute meaningfully,
indicating that stay-length policies and professional hosting behavior are related to pricing strategies.
Geographic variables (latitude and longitude) rank highly, supporting the strong role of location within NYC.
Overall, the feature importance results align with the EDA findings and provide an interpretable explanation for
why the non-linear model outperforms the linear baseline.

% ---------------- Discussion ----------------
\section{Discussion}
Results indicate that Airbnb pricing is influenced by complex interactions among location,
room type, availability, and host behavior. The superior performance of the Random Forest
model confirms the presence of non-linear pricing dynamics in the New York City Airbnb market.

% ---------------- Limitations ----------------
\section{Limitations and Future Work}
This study is limited by the absence of amenities, textual descriptions, and seasonal
effects. Future work could incorporate natural language processing, time-series analysis,
and classification-based pricing strategies.

% ---------------- Conclusion ----------------
\section{Conclusion}
This project demonstrates the application of data mining techniques to analyze Airbnb
pricing in New York City. Clustering reveals meaningful market segments, while regression
models highlight the importance of non-linear relationships in price prediction.

% ---------------- Tools Used ----------------
\section{Tools Used}

The following tools and technologies were used in this project:

\begin{itemize}
\item Python 3.12
\item Jupyter Notebook (Anaconda distribution)
\item pandas and NumPy for data manipulation
\item matplotlib and seaborn for visualization
\item scikit-learn for machine learning and model evaluation
\item LaTeX (Overleaf) for report preparation
\item Google Slides for project presentation
\end{itemize}

% ---------------- References ----------------

\section{References}
\renewcommand{\refname}{}
\vspace{-2.75em}
\begin{thebibliography}{9}
\bibitem{insideairbnb}
Inside Airbnb. \textit{New York City Airbnb Data}.  
Available at: \url{https://insideairbnb.com/get-the-data/}

\bibitem{sklearn}
Scikit-learn Documentation.  
\url{https://scikit-learn.org}

\end{thebibliography}

\end{document}